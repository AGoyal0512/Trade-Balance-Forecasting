\documentclass[12pt]{article}
\usepackage[utf8]{inputenc}
\usepackage[a4paper,top=4cm,bottom=2cm,left=2.5cm,right=2.5cm,marginparwidth=3cm]{geometry}
\usepackage{biblatex}
\usepackage{appendix}
\addbibresource{references.bib}

\title{\vspace{-3cm}Forecasting Trade Balance in Goods and Services\\ for the United States}
\bigskip
\bigskip
\bigskip
\author{
  \textbf{Agam Goyal}\\
  \texttt{agoyal25@wisc.edu}
  \and
  \textbf{Malkom Castellanos}\\
  \texttt{mcastellano3@wisc.edu}
}
\bigskip
\bigskip
\bigskip
\date{May 3, 2022}

\begin{document}

\maketitle

\section{Introduction}

The economic variable that we chose to forecast is the U.S. International Trade Balance in Goods and Services. International Trade Balance in Goods and Services attempts to accurately measure the trade balance of the United States, which is the difference between imports and exports measured in the millions of US dollars (\$). 

This economic time series also captures what the top exports and imports in the United States were in a given month. However, for the purpose of the forecast, we will be looking at the value of the Trade Balance for the coming 12 months. To provide a more complete picture of the forecasts, we look at both point and interval forecasts.

\section{Data}

The original publisher of the data we use is the Bureau of Economic Analysis (BEA). However, we obtained the data from the Federal Reserve Economic Data (FRED)\cite{data} using the FRED code \texttt{BOPGSTB}.

\section{Methods}

\section{Models}

\section{Forecasts and Discussions}

\begin{center}
        \printbibliography %Prints bibliography
\end{center}

\appendixname{ A}

\end{document}
